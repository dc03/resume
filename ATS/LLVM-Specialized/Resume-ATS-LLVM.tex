\documentclass[12pt]{article}

\usepackage{enumitem}
\usepackage{geometry}
\usepackage{makecell}
\usepackage{multicol}
\usepackage{setspace}
\usepackage{soul}
\usepackage{tabularx}
\usepackage{xcolor}

\usepackage{titlesec}
\titleformat{\section}
  {\normalfont\large\color{\primarycolor}\vspace{-3.2ex}}{}{0ex}{}

\newgeometry{
    top=10mm,
    bottom=10mm,
    left=10mm,
    right=10mm,
}

\usepackage{hyperref}
\hypersetup{
	colorlinks=true,
    linkcolor=red,
    linkbordercolor=red,
    filecolor=red,
    urlcolor=black,
    urlbordercolor=black,
    breaklinks=true,
}

\makeatletter
\Hy@AtBeginDocument{
  \def\@pdfborder{0 0 1}
  \def\@pdfborderstyle{/S/U/W 1}
}
\makeatother

\newcommand*\justify{%
  \fontdimen2\font=0.4em% interword space
  \fontdimen3\font=0.2em% interword stretch
  \fontdimen4\font=0.1em% interword shrink
  \fontdimen7\font=0.1em% extra space
  \hyphenchar\font=`\-% allowing hyphenation
}

\newcommand{\primarycolor}{red}
\newcommand{\mysection}[1]{\section{#1}\vspace{-1.2ex}}
\newcommand{\myhspace}[1]{\hspace*{\fill}{#1}}
\newcommand{\myhrule}[1]{
\vspace{1ex}
\hrule
\vspace{#1}
}

\begin{document}
    \pagenumbering{gobble}
    \hrule
    \vspace{2ex}

    \begin{center}
        \begin{tabularx}{\textwidth}{>{\raggedright\arraybackslash}X
                                     r}
            \textbf{\Huge{Dhruv Chawla}} \vspace{1ex} &
            {\LARGE{\color{\primarycolor}Compiler Engineer}} \\
        \end{tabularx}

        \vspace{1ex}
        \begin{tabularx}{\textwidth}{>{\raggedright\arraybackslash}X
                         p{0.2\textwidth}
                         c
                         >{\raggedleft\arraybackslash}X}
            {\href{mailto:dhruv263.dc@gmail.com}{dhruv263.dc@gmail.com}} &
            {+91 9910299843} &
            {\url{https://dc03.github.io}} &
            {Vellore, Tamil Nadu} \\
        \end{tabularx}
    \end{center}

    \vspace{-1.2ex}
    \myhrule{2ex}
    {
        \noindent A capable programmer learning compiler design and the LLVM compiler infrastructure. Experienced with interpreters, and familiar with systems and assembly programming. 4 years of C++ experience.
    }
    \vspace{0.8ex}
    \myhrule{2ex}

    \mysection{Languages}
    {
        \vspace{-4ex}
        \begin{itemize}[leftmargin=20ex,itemsep=0ex]
            \item English - Fluent
            \item Hindi - Intermediate
        \end{itemize}
        \vspace{-1.5ex}
    }
    \myhrule{2ex}

    \mysection{Skills}
    {
        \vspace{-3ex}
        \begin{center}
            \begin{multicols}{5}
                \begin{itemize}[nosep,parsep=0.1ex,itemsep=0.1ex]
                    \item C++
                    \item C
                    \item Python
                    \item Rust
                    \item Compilers
                    \item Interpreters
                    \item Linux
                    \item Git
                    \item LLVM
                \end{itemize}
            \end{multicols}
        \end{center}
        \vspace{-1.5ex}
    }
    \myhrule{2ex}

    \mysection{Experience}
    {
        % \renewcommand{\labelitemi}{}
        % \renewcommand{\labelitemii}{}
        % \renewcommand{\labelitemiii}{}
        \begin{itemize}
            \item
                \textbf{Google Summer of Code Contributor (The LLVM Compiler Infrastructure)}

                {\color{\primarycolor}May 29, 2023 - September 25, 2023}

                \textit{Project name: Improving Compile Times}
                \vspace{-1ex}
                \begin{itemize}[itemsep=1ex, leftmargin=3.5ex]
                    \item[-] Implemented the InferAlignment pass into the LLVM optimization pipeline
                    \item[-] Added a new diagnostic to clang
                    \item[-] Implemented transforms within InstCombine
                    \item[-] Found miscellaneous performance improvements in DAGCombiner and InstCombine
                    \item[-] Found a major performance improvement in SetVector
                    \item[-] \url{https://summerofcode.withgoogle.com/programs/2023/projects/JdqGUwNq}
                \end{itemize}
            \item
                \textbf{Google Summer of Code Contributor (The ENIGMA Team)}

                {\color{\primarycolor}June 13, 2022 - September 12, 2022}

                \textit{Project name: Data Buffers / Serialization}
                \vspace{-1ex}
                \begin{itemize}[itemsep=1ex, leftmargin=3.5ex]
                    \item[-] Worked on rewriting most of the frontend of the ENIGMA Development Language compiler, a scripting language based on GML
                    \item[-] Rewrote most of the Binary Buffer system which deals with storing and reading data from byte streams
                    \item[-] Made a serialization and deserialization system which uses template metaprogramming for static polymorphism
                    \item[-] \url{https://summerofcode.withgoogle.com/programs/2022/projects/BrXiUNA2}
                \end{itemize}
        \end{itemize}
    }
    \myhrule{2ex}

    \pagebreak

    \myhrule{2ex}
    \mysection{Education}
    {
        \begin{itemize}[itemsep=0.5ex,rightmargin=1ex]
            \item
                \textbf{B.Tech in Information Technology}
                \myhspace{\textit{VIT, Vellore} | 2020 - 2024}

                Current CGPA: 9.28

            \item
                \textbf{XIIth Grade (Senior Secondary), CBSE}
                \myhspace{\textit{Navy Children School, Mumbai} | 2020}

                Percentage: 96.4\%

            \item
                \textbf{Xth Grade (Secondary), CBSE}
                \myhspace{\textit{Navy Children School, Mumbai} | 2018}

                Percentage: 93.6\%
        \end{itemize}
    }
    \myhrule{2ex}

    \mysection{Projects}
    {
            \begin{itemize}[itemsep=0.8ex]
                \item
                    \textbf{nyx}\myhspace{\url{https://github.com/dc03/nyx} | \textit{September 2020}}
                    \begin{itemize}[itemsep=0.5ex, leftmargin=3.5ex]
                        \item[-] A simple, interpreted language implemented in C++
                        \item[-] Features classes with constructors and destructors, lists, tuples
                        \item[-] Static type system
                        \item[-] Copy, reference and move semantics
                        \item[-] Bytecode virtual machine
                        \item[-] Code formatter, bytecode dumper, VM execution tracing
                    \end{itemize}
                \item
                    \textbf{rispy}\myhspace{\url{https://github.com/dc03/rispy} | \textit{February 2022}}
                    \begin{itemize}[itemsep=0.5ex, leftmargin=3.5ex]
                        \item[-] Interpreter for a lispy-inspired lisp
                        \item[-] Implemented in Rust
                        \item[-] Tree-walk interpreter
                        \item[-] Testing for lexical analyzer and parser
                    \end{itemize}
                \item
                    \textbf{tictactoe-arduino}\myhspace{\url{https://github.com/dc03/tictactoe-arduino} | \textit{February 2023}}
                    \begin{itemize}[itemsep=0.5ex, leftmargin=3.5ex]
                        \item[-] Tic-tac-toe implemented on an Arduino Uno
                        \item[-] Multiplexing of outputs (LEDs) and inputs (buttons) to reduce pin usage
                        \item[-] Compact layout of game state to reduce memory usage
                        \item[-] Part of a university course project
                    \end{itemize}
            \end{itemize} 
    }
    \myhrule{2ex}

    \mysection{Certifications}
    {
        \begin{itemize}
            \item
                \textbf{Introduction to Haskell Programming (NPTEL)}\myhspace{\textit{Issued Sep 2022}}

                {Percentage: 85\%\myhspace{\href{https://archive.nptel.ac.in/noc/Ecertificate/?q=NPTEL22CS69S2318078809012045}{Credential ID NPTEL22CS69S2318078809012045}}}
            \item
                \textbf{Compiler Design (NPTEL)}\myhspace{\textit{Issued Apr 2022}}

                {Percentage: 90\%\myhspace{\href{https://archive.nptel.ac.in/noc/Ecertificate/?q=NPTEL22CS14S2446142802071248}{Credential ID NPTEL22CS14S2446142802071248}}}
            \item
                \textbf{Design and Analysis of Algorithms (NPTEL)}\myhspace{\textit{Issued Oct 2021}}

                {Percentage: 85\%\myhspace{\href{https://archive.nptel.ac.in/noc/Ecertificate/?q=NPTEL21CS68S4332059403122958}{Credential ID NPTEL21CS68S4332059403122958}}}
        \end{itemize}
    }
    \myhrule{2ex}

    \begin{center}
        \href{https://github.com/dc03/resume}{Made with \LaTeX}
    \end{center}
\end{document}
